%-------------------------
% Resume in Latex
% Based off of : https://github.com/sb2nov/resume/
% License : MIT
%------------------------

\documentclass[letterpaper,11pt]{article}
\usepackage{latexsym}
\usepackage[empty]{fullpage}
\usepackage{titlesec}
\usepackage{marvosym}
\usepackage[usenames,dvipsnames]{color}
\usepackage{verbatim}
\usepackage{enumitem}
\usepackage[hidelinks]{hyperref}
\usepackage{fancyhdr}
\usepackage[english]{babel}
\usepackage{tabularx}
\usepackage{setspace}
\usepackage{geometry}
\input{glyphtounicode}

\pagestyle{fancy}
\fancyhf{} % clear all header and footer fields
\fancyfoot{}
\renewcommand{\headrulewidth}{0pt}
\renewcommand{\footrulewidth}{0pt}

\geometry{
    top=0.5in,
    left=0.5in,
    right=0.5in,
    bottom=0.5in
}
\setstretch{1.1}

\urlstyle{same}

\raggedbottom
\raggedright
\setlength{\tabcolsep}{0in}

% Sections formatting
\titleformat{\section}{
  \vspace{-12pt}\scshape\raggedright\Large
}{}{1pt}{}[\color{black}\titlerule \vspace{-6pt}]

% Ensure that generate pdf is machine readable/ATS parsable
\pdfgentounicode=1
\newcommand{\listitem}[1]{
  \item \small #1
}

% Just in case someone needs a heading that does not need to be in a list
\newcommand{\resumeHeading}[4]{
    \begin{tabular*}{0.99\textwidth}[t]{l@{\extracolsep{\fill}}r}
      \textbf{#1} & #2 \\
      \textit{\small#3} & \textit{\small #4} \\
    \end{tabular*}
}

\newcommand{\titleplacehead}[2]{
  \item\begin{tabular*}{0.97\textwidth}[t]{l@{\extracolsep{\fill}}r}
      \textbf{#1} & \textbf{#2}
    \end{tabular*}
}

\newcommand{\roletimehead}[2]{
    \item\begin{tabular*}{0.97\textwidth}{l@{\extracolsep{\fill}}r}
      #1 & \textit{#2} \\
    \end{tabular*}
}

\newcommand{\titletimehead}[2]{
    \item\begin{tabular*}{0.97\textwidth}{l@{\extracolsep{\fill}}r}
      \textbf{#1} & \textit{#2} \\
    \end{tabular*}
}

\newcommand{\itemdesc}[2]{\item\small\textbf{#1:}{#2}}

\renewcommand{\labelitemii}{$\vcenter{\hbox{\tiny$\bullet$}}$}

\newcommand{\headstart}{\begin{itemize}[leftmargin=0in, label={}, itemsep=-5pt]}
\newcommand{\headend}{\end{itemize}}
\newcommand{\itemstart}{\vspace{-5pt}\begin{itemize}[rightmargin=0.2in, itemsep=-1pt]}
\newcommand{\itemend}{\end{itemize}}

\begin{document}

%----------HEADING-----------------
\begin{center}
  \textbf{{\LARGE Neo Lee}} \\
  \href{mailto:neo.lky852@gmail.com}{neo.lky852@gmail.com} $\vert$
  \href{tel:+12137130311}{(213) 713-0311} $\vert$
  \href{https://github.com/moneybabe}{github.com/moneybabe}
\end{center}


%-----------EDUCATION-----------------
\section{Education}
  \headstart
    \titleplacehead
      {University of California, Berkeley}{GPA: 4.0/4.0}
    \roletimehead
      {Bachelor of Arts, Applied Mathematics, Computer Science}{Graduation: Spring 2025}
    \vspace{2pt}
    \listitem{Cal Alumni Leadership Scholarship}
    \vspace{5pt}
    \itemdesc{Relevant Coursework}{
      Time Series Analysis,
      Stochastic Processes,
      Probability Theory,
      Linear Algebra,
      Discrete Mathematics,
      Graph Theory,
      Real Analysis,
      Numerical Analysis,
      Data Structures and Algorithms,
      Functional Programming,
      Object Oriented Programming,
      Dynamic Programming,
      Cryptography
    }
  \headend


%-----------EXPERIENCE-----------------
\section{Experience}
  \headstart

    \titleplacehead
    {UC Berkeley Department of Mathematics}{Berkeley, CA}
      \roletimehead
        {Undergraduate Researcher - Stake-governed Random Turn Games}{August 2023 - Present}
        \itemstart
          \listitem{
            Built a finite integer line tug-of-war game simulator with Python, Numpy, and Pandas to 
            solve for Markov perfect equilibria with dynamic programming, and visualized the 
            results with Matplotlib.
          }
          \listitem{
            Constructed a computer assisted proof for the sufficient and necessary condition for
            the existence of a Markov perfect equilibrium in infinite integer line tug-of-war games, 
            being that the reward ratio is bounded within a $1\times10^{-4}$ interval from a symmetric 
            game: paper is currently under review.
          }
          \listitem{
            Reduced the run-time of the computer-assisted proof by 60\% through the 
            implementation of dynamic programming optimization techniques.
          }
        \itemend
        
      \roletimehead
        {Undergraduate Researcher - Mechanistic Interpretability}{September 2023 - Present}
        \itemstart
          \listitem{
            Reverse engineered Stockfish's efficiently updatable neural network's learned algorithm 
            using Pytorch, Sklearn, and Seaborn, achieving MSE of 1.8 compared to a simple linear 
            regression model with MSE of 126.
          }
          \listitem{
            Applied dimension reduction techniques such as SVD, neuron pruning, and feature 
            projection onto ReLU privileged basis to reverse engineer the embedding layer, showing 
            that each neuron's activation space is 95\%+ correlated.
          }
          \listitem{
            Aggregated 38GB training dataset with Sqlite and used Git for version 
            control.
          }
          \listitem{
            Built an Alpha-beta pruning algorithm \& NN based chess engine with Python and C++ to 
            study the effect of neural network based evaluation functions on the performance of the 
            algorithm.
          }
        \itemend

    \titleplacehead
      {UC Berkeley Department of EECS}{Berkeley, CA}
      \roletimehead{Academic Tutor - CS61A}{August 2023 - Present}
        \itemstart
          \listitem{
            Tutored students in Functional Programming, Object Oriented Programming, 
            and Dynamic Programming with Python in lab sessions.
          }
          \listitem{
            Held weekly office hours to help students with homework and projects, and tutored 
            other tutors.
          }
        \itemend
  \headend


%-----------PROJECTS-----------------
\section{Projects}
  \headstart
    \titletimehead{Mathematical Error Analysis Library}{December 2023 - Present}
      \itemstart
        \listitem{
          Building a Numpy-like Python library to perform arithmetic
          operations object-orientedly with associated error bounds factoring in floating point 
          error.
        }
        \listitem{
          Can be used in rigorous computer assisted proofs to bound the error of the results of
          mathematical expressions.
        }
      \itemend
    
    \titletimehead{2D Tile-based World Exploration Engine}{November 2023 (2 days)}
      \itemstart
        \listitem{
          Built a 2D tile-based world exploration engine with Java that generates a random world 
          with rooms and corridors, which allows the user to explore the world with a character, 
          and was showcased for class project demo with an A+.
        }
        \listitem{
          Implemented the A* search algorithm to find the shortest path between two points in the 
          world and a snake game.
        }
      \itemend

    \titletimehead{Trading Bot}{June 2023 - August 2023}
      \itemstart
        \listitem{
          Implemented machine learning models with Pytorch, Sklearn, and Statsmodels for 
          cryptocurrency price forecasting, e.g. LSTM, ARIMA, ETS Smoothing, Multi-linear Regression, 
          Random Forest, Sentiment Analysis, Transformer.
        }
        \listitem{
          Built a trading bot with Python to trade cryptocurrency on Bybit using the Bybit API, 
          achieving 2.68 Sharpe Ratio on the past 3 years of data by performing time series 
          cross validation and leading factors engineering.
        }
      \itemend

    \titletimehead{Interview Questions Scraper}{April 2023 (1 hour)}
      \itemstart
        \listitem{
          Built a web scraper with Python and Selenium to scrape interview questions from 
          Glassdoor.
        }
      \itemend
  \headend

\end{document}
